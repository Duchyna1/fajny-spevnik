\begin{song}[
    remember-chords = true ,
    verse/numbered = true ,
    transpose-capo = true
  ]{
    title = Die Ableitung von Sinus,
    band = DorFuchs,
    %capo = 1 ,
    key  = Fmi
  }
	\newversetype{r1}[name=R1]
	\newversetype{r2}[name=R2]


    \begin{r1}
    ^{G}Die Ableitung ^{D}vom Sinus ist der ^{Em}Kosinus ^{C} \\
    Und die Ableitung vom Kosinus ist minus Sinus \\
    Und die Ableitung davon ist dann minus der Kosinus \\
    Und davon ist die Ableitung der Sinus
    \end{r1}

    \begin{r1}
    \end{r1}

    \begin{verse}
    ^{E}Wenn wir uns die Sinuskurve mal genau ansehn \\
    ^{E}Dann können wir einfach durch drauf schauen verstehn \\
    ^{D}Was die Ableitung ist, also schauen wir uns das an \\
    ^{E}Der Sinus fängt ja hier bei 0 erstmal an \\
    ^{G}Und steigt dann in einem 45-Grad-Winkel an \\
    ^{E}Weshalb wir an der Stelle 0 hier Anstieg 1 haben \\
    ^{E}Und danach nimmt die Steigung langsam ab \\
    ^{G}Bis ich bei Pi Halbe einen Hochpunkt hab \\
    ^{A}Wo die Steigung an dem Punkt selbst 0 beträgt \\
    ^{G}Und danach sieht man ja, dass es nach unten geht \\
    ^{H}Und die steilste Stelle ist genau bei Pi \\
    ^{G}Mit Anstieg -1, denn wenn man genau hinsieht \\
    Ist hier ne Symmetrie zu den 45 Grad \\
    Die ich am Anfang nach oben und jetzt nach unten hab \\
    Und dann flacht das wieder ab, bis ich bei 3 Halbe Pi \\
    Einen Tiefpunkt erreiche und dann ist 0 der Anstieg \\
    Und dann geht es nach oben bis ich die x-Achse erreiche \\
    Und ab diesem Punkt wird es wie am Anfang das Gleiche: \\
    Anstieg 1 und dann 0 und -1 und so weiter \\
    Das geht dann ja periodisch immer weiter \\
    Und die Ableitung vom Sinus ist jetzt jene Funktion \\
    Die diese Werte annimmt und vielleicht kennst du die ja schon \\
    Das ist der Kosinus, wodurch du weißt \\
    Warum es in diesem Lied hier heißt:
    \end{verse}

    \begin{r1}
    ($\times2$)
    \end{r1}

    \begin{bridge}
    ^{Em7}Und jetzt können ^{Dm7}wir nicht nur differenzier^{Cma}en, sondern ^{Ebmaj7}auch ^{F7}integrieren
    \end{bridge}

    \begin{r2}
    Ne ^{Bb}Stammfunktion ^{F}vom Kosinus ist ^{Gm}Sinus ^{Eb}\\
    Und ne Stammfunktion vom Sinus ist minus Kosinus \\
    Und ne Stammfunktion davon ist dann minus der Sinus \\
    Und davon ist ne Stammfunktion der Kosinus
    \end{r2}

 	\begin{r2}
 	\end{r2}

\end{song}